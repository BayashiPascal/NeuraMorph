\section*{Introduction}

NeuraMorph is a C library providing structures and functions to implement a neural network.\\ 

It uses the \begin{ttfamily}PBErr\end{ttfamily}, \begin{ttfamily}PBMath\end{ttfamily}, \begin{ttfamily}GSet\end{ttfamily} library.\\

\section{Definitions}

\section{Interface}

\begin{scriptsize}
\begin{ttfamily}
\verbatiminput{/home/bayashi/GitHub/NeuraMorph/neuramorph.h}
\end{ttfamily}
\end{scriptsize}

\section{Code}

\subsection{neuramorph.c}

\begin{scriptsize}
\begin{ttfamily}
\verbatiminput{/home/bayashi/GitHub/NeuraMorph/neuramorph.c}
\end{ttfamily}
\end{scriptsize}

\subsection{neuramorph-inline.c}

\begin{scriptsize}
\begin{ttfamily}
\verbatiminput{/home/bayashi/GitHub/NeuraMorph/neuramorph-inline.c}
\end{ttfamily}
\end{scriptsize}

\section{Makefile}

\begin{scriptsize}
\begin{ttfamily}
\verbatiminput{/home/bayashi/GitHub/NeuraMorph/Makefile}
\end{ttfamily}
\end{scriptsize}

\section{Unit tests}

\begin{scriptsize}
\begin{ttfamily}
\verbatiminput{/home/bayashi/GitHub/NeuraMorph/main.c}
\end{ttfamily}
\end{scriptsize}

\section{Unit tests output}

\begin{scriptsize}
\begin{ttfamily}
\verbatiminput{/home/bayashi/GitHub/NeuraMorph/unitTestRef.txt}
\end{ttfamily}
\end{scriptsize}

